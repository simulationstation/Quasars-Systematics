%% MNRAS letter -- CatWISE quasar dipole systematics investigation
%% Figures expected in this folder:
%%   baseline_rvmp_fig5_poisson_glm.png
%%   cmb_projection_plot.png
%%   validate_depth_systematic_recovery.png
%%   D_vs_epoch_compare.png
%%   N_vs_epoch.png
%%   drift_mc_null_hist.png
%%   drift_mc_inject_bias.png
%%   lss_cov_D_hist_w1max16p6.png
%% Additional figures and methodological details are referenced as Supplementary Data.

\documentclass[fleqn,usenatbib]{mnras}

\usepackage[T1]{fontenc}
\usepackage{amsmath}
\usepackage{txfonts}
\usepackage{graphicx}
\usepackage{bm}
\usepackage{lastpage}

\title[Systematics in the CatWISE quasar dipole]{Time-domain instability and selection systematics in the CatWISE quasar dipole}

\author[A. Smith]{
Aiden Smith$^{1}$\thanks{E-mail: aidenblakesmithtravel@gmail.com} \\
$^{1}$Independent researcher
}

\date{Accepted XXX. Received YYY; in original form ZZZ}
\pubyear{2026}

\begin{document}
\label{firstpage}
\pagerange{\pageref{firstpage}--\pageref{LastPage}}
\maketitle

\begin{abstract}
Recent studies have reported a dipole in CatWISE AGN number counts with an amplitude that exceeds standard kinematic expectations.
We investigate this signal using the publicly available CatWISE sample and exclusion mask.
By scanning cumulative faint limits $W1 \le W1_{\rm max}$ ($15.5 \le W1_{\rm max} \le 16.6$) on a fixed, coverage-defined footprint, we fit dipoles using both linear regression, following the methodology of \citet{secrest2022} after removing the dominant ecliptic-latitude trend, and Poisson generalised linear models (GLM) on HEALPix counts with nuisance templates.
While the dipole amplitude remains stable near $D \simeq 1.6 \times 10^{-2}$, the dipole axis is unstable: its sign-invariant separation from the CMB dipole axis increases from $\sim1^\circ$ at $W1_{\rm max}=15.5$ to $\sim34^\circ$ at $W1_{\rm max}=16.6$, driven by a growing component perpendicular to the CMB.
Clustered lognormal mock catalogues at $W1_{\rm max}=16.6$ yield $D_{\rm hat} \approx 0.0168$ with $\sigma_D \approx 0.0023$ (S/N $\approx 7$).
Injection and recovery tests demonstrate that modest depth-linked completeness modulations can bias both amplitude and axis when the depth model is misspecified.
Furthermore, a correlated-cut Monte Carlo analysis shows that the observed drift span lies in the $<1$ per cent tail under a Poisson-only null hypothesis, while injected ecliptic-longitude patterns can drive per cent-level dipoles if longitude templates are omitted.
Finally, an epoch-sliced unWISE time-domain analysis (2010--2020) restricted to the Secrest-accepted CatWISE parent sample yields apparently epoch-dependent dipole amplitudes under dipole-only recovery ($D\simeq0.067$--$0.118$).
However, a constrained rich parametric bootstrap with a shared dipole vector across epochs gives $P(\Delta D_{\rm sim}\ge\Delta D_{\rm obs})=0.974$ and $P(\chi^2_{\rm sim}\ge\chi^2_{\rm obs})=0.983$, showing the observed variability is expected under a constant physical dipole plus measured epoch-dependent systematics.
We therefore treat epoch slicing as a systematics-generability diagnostic rather than evidence for a time-varying physical dipole.
\end{abstract}

\begin{keywords}
quasars: general -- surveys -- catalogues -- methods: statistical -- cosmology: observations
\end{keywords}

\section{Introduction}
A kinematic dipole resulting from Solar System motion is expected in number counts; however, sufficiently large-scale tracers should otherwise be statistically isotropic.
\citet{secrest2022} reported a dipole in CatWISE AGN candidates with an amplitude significantly larger than theoretical kinematic expectations.
Flux-limited dipole measurements can be sensitive to completeness and scan-depth structure near the faint selection boundary.
Therefore, a central question regarding the robustness of these results is whether the inferred solution, particularly the axis direction, is stable under plausible changes to the faint limit and completeness or scan modelling.

In this paper, we focus on six results that most directly diagnose the control of selection effects and systematics on the inference: (i) faint-limit axis drift and its decomposition into CMB-parallel and perpendicular components; (ii) an end-to-end depth-linked injection and recovery demonstration; (iii) a clustered-mock-covariance verification for amplitude signal-to-noise (S/N); (iv) a correlated-cut Monte Carlo demonstrating how scan- or season-linked longitude structure can mimic per cent-level dipoles; (v) a time-domain stability test using the unWISE time-domain catalogue; and (vi) a constrained rich parametric bootstrap that calibrates epoch-variability statistics under a constant dipole plus epoch-dependent nuisance fields.
Full methodological details, additional figures, and extended robustness suites are provided as Supplementary Data on Zenodo.

\section{Data and methodology}
\subsection{CatWISE sample, masking, and fixed footprint}
We utilise the publicly released CatWISE AGN candidate sample and the associated exclusion mask \citep{catwise2020,secrest2022}.
Following the published baseline criteria, we impose $|b|>30^\circ$ and require $W1_{\rm cov}\ge 80$.
We define cumulative samples by a hard faint limit $W1 \le W1_{\rm max}$ over the range $W1_{\rm max}\in[15.5,16.6]$.

To isolate magnitude-limit dependence from footprint changes, we enforce a fixed footprint across the scan: the valid HEALPix pixel set is defined strictly from coverage and exclusion products and the parent $W1_{\rm cov}\ge 80$ requirement.
This construction is independent of shot-noise or data fluctuations at each $W1_{\rm max}$.

\subsection{Dipole estimators and axis diagnostics}
We implement two primary estimators: (i) a linear regression following the methodology of \citet{secrest2022}, performed after removing the dominant ecliptic-latitude trend in the mean density, and (ii) Poisson maximum-likelihood fits on HEALPix counts via a GLM with a log link function:
\begin{equation}
\lambda(\hat n)=\exp\!\left[\beta_0+\bm{\beta}_{\rm dip}\!\cdot\!\hat n+\sum_{k}\beta_k\,t_k(\hat n)\right],
\end{equation}
where $t_k$ represent nuisance templates (e.g.\ depth and scan proxies), and $D\simeq|\bm{\beta}_{\rm dip}|$ for small amplitudes.
For drift comparisons, we quote the sign-invariant axis separation
$\Delta\theta_{\rm axis}\equiv\cos^{-1}\!\left(|\hat{\bm d}_1\!\cdot\!\hat{\bm d}_2|\right)$.
To interpret drift relative to the CMB, we decompose $\bm{\beta}_{\rm dip}$ into CMB-parallel and perpendicular components.

\subsection{Uncertainty modelling, injections, and drift tests}
We use Poisson resampling for scan-point uncertainties, and clustered lognormal mock catalogues to provide a diagnostic large-scale-structure covariance for $D$ at a representative faint cut.
We perform end-to-end injection and recovery tests to demonstrate depth-template misspecification biases.
To address concerns about correlated nested cuts and scan/season effects, we run a \emph{correlated-cut drift Monte Carlo} that simulates independent Poisson counts in \emph{differential} $W1$ bins and cumulatively sums them to form correlated $W1<{\rm cut}$ maps (preserving the dependence structure of real scans).
We also include a longitude-pattern injection test to show how unmodelled ecliptic-longitude structure can inflate dipole estimates.

Low-$\ell$ mask coupling is a recurrent concern in masked-sky analyses.
In Supplementary Data, we further validate a GLM extension that marginalises $\ell\ge2$ spherical harmonic nuisance modes with a physically motivated Gaussian prior calibrated from clustered mock catalogues; a masked-sky injection test shows that unregularised harmonics inflate uncertainty while the prior-regularised fit remains stable.

\subsection{Time-domain epoch-sliced test (CatWISE parent)}
A kinematic/cosmological dipole is time-invariant.
As a diagnostic directly tied to the published CatWISE selection, we restrict to the Secrest-accepted CatWISE parent sample on our fixed footprint ($W1\le16.4$, $W1-W2\ge0.8$, $W1_{\rm cov}\ge80$, $|b|>30^\circ$ and the exclusion mask).
For each unWISE time-domain epoch \citep{unwise_lang2014,unwise_meisner2017}, we match parent objects to epoch photometry within $2^{\prime\prime}$ and form epoch-sliced HEALPix count maps using uniform per-epoch quality cuts (${\tt primary}=1$, ${\tt flags\_unwise}=0$, ${\tt flags\_info}=0$, positive fluxes and errors).
We measure the dipole amplitude per epoch with (i) a dipole-only Poisson GLM, (ii) a minimal nuisance-basis GLM (dipole + $|\beta_{\rm ecl}|$, $\sin\lambda_{\rm ecl}$, $\cos\lambda_{\rm ecl}$, and $\log N_{\rm exp}$), and (iii) a vector-sum cross-check.
For the GLM-based fits we test a constant-amplitude null, $D_e=D_0$,
\begin{equation}
\bar D_w=\frac{\sum_e D_e/\sigma_e^2}{\sum_e 1/\sigma_e^2},\qquad
\chi^2_D=\sum_e\frac{(D_e-\bar D_w)^2}{\sigma_e^2},\qquad \nu=n_{\rm epoch}-1,
\end{equation}
where $\sigma_e$ is obtained from the per-epoch GLM covariance for $D=|\bm b|$ using the delta method,
$\mathrm{Var}(D)\simeq (\bm b/D)^\top \Sigma_{\bm b}(\bm b/D)$.
We use the diagonal-in-epoch approximation in this test.
Because this analytic statistic can understate variability under model mismatch, we also calibrate it with a constrained rich parametric bootstrap.
We fit
\begin{equation}
\log \mu_{e,p}=\alpha_e+\bm b_{\rm shared}\cdot\hat{\bm n}_p+\bm\gamma_e\cdot\bm T_p,
\end{equation}
with a shared dipole vector $\bm b_{\rm shared}$ across epochs and epoch-specific nuisance coefficients.

Here $\bm T_p$ includes one plausible extended set: $|\beta_{\rm ecl}|$, $\sin\lambda_{\rm ecl}$, $\cos\lambda_{\rm ecl}$, $\log N_{\rm exp}$, and the additional templates $E(B\!-\!V)$, $\log(1+N_{\rm star})$, and $\log{\rm invvar}$.
From this fitted null, we draw $N_{\rm sim}=1000$ Poisson realisations and recover each with the standard per-epoch dipole-only GLM, then compare $\Delta D$ and $\chi^2_D$ with the observed values.
We further run leave-one-template-out reruns in the rich basis to rank nuisance drivers of epoch variability.
Because the epoch maps are overlapping re-selections of a fixed parent list, this is a time-invariance diagnostic rather than a remeasurement of the all-epoch CatWISE dipole.

\section{Results}
\subsection{Stable amplitude but strong faint-limit axis drift}
Figure~\ref{fig:drift_and_cmb} shows the baseline Poisson GLM scan and the CMB decomposition.
The total amplitude remains near $D\simeq1.6\times10^{-2}$ across $W1_{\rm max}\in[15.5,16.6]$, but the best-fit axis drifts substantially with faint limit.
Quoting sign-invariant axis angles to the CMB dipole axis, the separation grows from $\sim1.3^\circ$ at $W1_{\rm max}=15.5$ to $\sim34^\circ$ at $W1_{\rm max}=16.6$.
The CMB-parallel/perpendicular decomposition shows that this drift is driven primarily by a growing non-CMB (perpendicular) component at fainter cuts, rather than by a simple reduction of a CMB-aligned component.

\begin{figure*}[t]
\centering
\includegraphics[width=0.49\textwidth]{baseline_rvmp_fig5_poisson_glm.png}\hfill
\includegraphics[width=0.49\textwidth]{cmb_projection_plot.png}
\caption{\textbf{Magnitude-limit drift and CMB decomposition.}
\textit{Left:} Baseline Poisson GLM scan on a fixed footprint, showing stable amplitude but a drifting axis as the faint limit is deepened.
\textit{Right:} Decomposition of the recovered dipole vector into CMB-parallel and CMB-perpendicular components; the drift is driven primarily by a growing perpendicular component at fainter cuts.}
\label{fig:drift_and_cmb}
\end{figure*}

\subsection{Clustered-mock-covariance verification: high amplitude S/N persists}
Figure~\ref{fig:lss_cov} shows the clustered lognormal mock-catalogue amplitude distribution at $W1_{\rm max}=16.6$.
For the real data we obtain $D_{\rm hat}\approx0.0168$ with mock percentiles consistent with $\sigma_D\approx0.0023$, implying an amplitude S/N of order $\sim7$ at this representative cut.
This covariance is used as a diagnostic upgrade beyond Poisson-only noise, not as a final precision covariance for cosmological parameter inference; we do not interpret the resulting S/N as a cosmological significance claim.

\begin{figure}[t]
\centering
\includegraphics[width=\columnwidth]{lss_cov_D_hist_w1max16p6.png}
\caption{Clustered lognormal mock amplitude distribution at $W1_{\rm max}=16.6$ (diagnostic LSS covariance). The real-data estimate lies near the centre of the mock distribution, implying $\sigma_D\approx 0.0023$ and amplitude S/N of order $\sim 7$.}
\label{fig:lss_cov}
\end{figure}

\subsection{End-to-end depth-linked injection and recovery}
Figure~\ref{fig:depth_injection} demonstrates that modest depth-linked selection systematics can bias both amplitude and axis when the fit omits the correct depth template, and that including the correct map-level depth covariate materially improves recovery.
This provides an explicit mechanism for the template sensitivity and axis instability observed in the real-data scans near the faint limit.

\begin{figure}[t]
\centering
\includegraphics[width=\columnwidth]{validate_depth_systematic_recovery.png}
\caption{End-to-end injection and recovery completeness validation. A depth-linked selection systematic biases both amplitude and axis when the fit is misspecified (omits the depth template), while including the correct depth-map covariate materially improves recovery.}
\label{fig:depth_injection}
\end{figure}

\subsection{Correlated-cut drift Monte Carlo and longitude-pattern injection}
Figure~\ref{fig:drift_mc} summarizes the correlated-cut drift Monte Carlo.
Under a correlated Poisson-only null (with a fixed injected dipole), the observed drift path length is typical, but the observed drift span (end-to-end and maximum pairwise separation across the scan) lies in the $\sim0.5$--$0.9$ per cent tail, indicating that the coherence of the observed migration is unlikely to be a random walk from Poisson fluctuations alone.
A seasonal/scan-linked longitude-pattern injection shows that if ecliptic-longitude structure is omitted from the recovery model, the inferred dipole can be inflated to the per cent level; adding $\sin \lambda$ and $\cos \lambda$ longitude templates restores a kinematic-scale injected dipole.

\begin{figure*}[t]
\centering
\includegraphics[width=0.49\textwidth]{drift_mc_null_hist.png}\hfill
\includegraphics[width=0.49\textwidth]{drift_mc_inject_bias.png}
\caption{\textbf{Correlated-cut drift Monte Carlo.}
\textit{Left:} Drift-metric distributions under a correlated Poisson-only null, with observed values marked; the drift span (end-to-end and max-pair separation) is in the $<1$ per cent tail.
\textit{Right:} Seasonal/scan-linked longitude-pattern injection: omitting longitude templates can inflate recovered dipoles to the per cent level, while including $\sin \lambda, \cos \lambda$ restores a kinematic-scale injected dipole.}
\label{fig:drift_mc}
\end{figure*}

\subsection{Controlled time-domain epoch-sliced diagnostic in unWISE (CatWISE parent)}
Figure~\ref{fig:time_domain} shows the epoch-sliced amplitudes and per-epoch sample sizes for the CatWISE-parent time-domain selection.
Across epochs 0--15 (excluding epoch 16, which has much smaller $N$ and is likely partial), dipole-only Poisson GLM fits give apparently epoch-dependent amplitudes $D=0.067$--$0.118$ (typical $\sigma_D\sim3$--$5\times10^{-3}$), while the vector-sum cross-check spans $0.058$--$0.155$ (expected broader; no nuisance marginalisation).
Under the constrained rich bootstrap null with a shared dipole vector ($D_{\rm shared}=0.0948$), the observed dipole-only variability is not extreme: $P(\Delta D_{\rm sim}\ge\Delta D_{\rm obs})=0.974$ and $P(\chi^2_{\rm sim}\ge\chi^2_{\rm obs})=0.983$ (equivalently, lower-tail probabilities $0.026$ and $0.017$).
The corresponding analytic $\chi^2_D$ statistic is large in dipole-only fits ($\chi^2=308.2$ for $\nu=15$, $p=1.10\times10^{-56}$) and in the minimal nuisance-basis fit ($\chi^2=51.0$ for $\nu=15$, $p=8.21\times10^{-6}$).
Using the richer nuisance basis defined in Section~2.4 yields a similarly large analytic result ($\chi^2=52.7$ for $\nu=15$, $p=4.32\times10^{-6}$).
Per-epoch comparability metrics are stable ($N=2.076$--$2.304\times10^5$; active-pixel fraction $0.974$--$0.997$), and the conclusion persists on common-support and matched-depth subsets (minimal basis: $\chi^2=46.5$, $p=4.50\times10^{-5}$; richer basis: $\chi^2=52.5$, $p=4.70\times10^{-6}$, both for $\nu=15$).
Any positive cross-epoch covariance would reduce the effective degrees of freedom of analytic $\chi^2_D$, further motivating the bootstrap calibration as the primary null.
Thus, large analytic $\chi^2_D$ values from dipole-only recovery are not a calibrated test for time-variation.
We then inject a constant true dipole through empirically fitted epoch-dependent nuisance fields and recover with dipole-only fits; in all $N_{\rm sim}=1000$ trials, both $\Delta D_{\rm sim}$ and $\chi^2_{\rm sim}$ exceed the observed values (equivalently, complementary Monte Carlo tail $<10^{-3}$ at this resolution).
This injection is intentionally anchored to empirically fitted epoch-dependent nuisance fields to test whether measured time-dependent survey structure can generate the observed spread under dipole-only recovery.
In this construction, $\chi^2_{\rm sim}$ is typically larger than $\chi^2_{\rm obs}$ because empirically fitted epoch-dependent nuisance fields can project strongly onto $\ell=1$ under dipole-only recovery.
Leave-one-template-out reruns rank the dominant drivers as $\cos\lambda_{\rm ecl}$ ($\Delta\chi^2=+22.4$), stellar-count ($+6.5$), $\sin\lambda_{\rm ecl}$ ($+4.4$), and depth ($+3.6$), indicating a primarily longitude-structured plus stellar/depth systematic origin.
The wider vector-sum range is expected because that cross-check does not marginalise nuisance templates; correspondingly, $(D_{\rm vec}-D_{\rm GLM})$ correlates with per-epoch nuisance-attribution strength ($r\simeq0.77$).
Here the null is constant amplitude $D_e=D_0$; direction stability is reported separately in Supplementary Data.

\begin{figure*}[t]
\centering
\includegraphics[width=0.49\textwidth]{D_vs_epoch_compare.png}\hfill
\includegraphics[width=0.49\textwidth]{N_vs_epoch.png}
\caption{\textbf{Controlled time-domain epoch-sliced diagnostic (unWISE; CatWISE parent).}
\textit{Left:} Epoch-resolved dipole amplitudes for epochs 0--15 from a dipole-only Poisson GLM fit and a vector-sum cross-check; amplitudes vary across epoch under dipole-only recovery.
\textit{Right:} Per-epoch sample sizes for the fixed CatWISE parent footprint; epoch 16 has much smaller $N$ and is likely partial.}
\label{fig:time_domain}
\end{figure*}

\section{Discussion and conclusions}
Across multiple estimators on a fixed footprint, the CatWISE number-count dipole remains non-zero at the $D\sim10^{-2}$ level, and clustered mock catalogues imply a high nominal amplitude S/N in this diagnostic covariance treatment.
However, the axis drifts with the faint limit, driven by a growing CMB-perpendicular component, suggesting a degeneracy with completeness or scan modelling near the magnitude boundary.
End-to-end injections and correlated-cut Monte Carlo tests show that modest depth- and scan-linked modulations, including ecliptic-longitude structure, can bias both amplitude and axis when the recovery model is misspecified.
Finally, unWISE epoch slicing restricted to the Secrest-accepted CatWISE parent sample yields apparently time-dependent dipole-only amplitudes.
Although analytic constant-$D$ statistics are large, richer-template injections and constrained rich parametric bootstrap calibration show that the observed epoch variability is expected under a constant shared dipole once measured epoch-dependent systematics are propagated through dipole-only recovery.
We therefore do not interpret epoch variability as evidence for a time-varying physical dipole; instead it indicates that dipole-only epoch fits are systematics-dominated.
In Supplementary Data, we also run a maximal-nuisance GLM suite at a representative cut ($W1_{\rm max}=16.6$) with a ridge-regularised extended template basis and held-out sky validation.
The free-axis solution is strongly nuisance-subspace sensitive: a baseline fit gives $D=0.01678$ with axis separation $34.3^\circ$ to the CMB, while a maximal-nuisance fit gives $D=0.00578$ and $73.4^\circ$; constraining nuisance templates to be orthogonal to the monopole/dipole subspace restores $D=0.01776$ and $34.8^\circ$, illustrating strong low-$\ell$ coupling under the footprint.
In the CMB-fixed model, we find $D_{\parallel}=-0.00815$ in-sample and $D_{\parallel,{\rm test}}=-0.00250$ in a 2-fold held-out wedge split.
A constrained-null parametric bootstrap with $D_{\parallel,{\rm true}}=0.0046$ yields $p_{\rm abs}\equiv P(|D_{\parallel,{\rm sim}}|\ge |D_{\parallel,{\rm obs}}|)=0.707$, indicating that the observed CMB-parallel amplitude is typical under a kinematic-scale dipole once measured systematics are propagated through the recovery pipeline.
Leave-one-template-out attribution ranks the dominant drivers as explicit low-$\ell$ harmonic modes (e.g.\ $Y_{2,0}$ and $Y_{2,1}$), stellar-density, and ecliptic-longitude structure, reinforcing the role of mode coupling and selection systematics.
These diagnostics motivate caution regarding cosmological interpretations of the CatWISE dipole without an end-to-end completeness model validated against external information and in the time domain.

\clearpage

\noindent\textit{Data and code availability:} Software and reproducibility materials for this analysis are archived on Zenodo at \doi{10.5281/zenodo.18643926}; related supplementary assets used in this study are archived at \doi{10.5281/zenodo.18489200}.
This work used the CatWISE2020 catalogue and the unWISE Time-Domain catalogue provided via NASA/IPAC IRSA (\doi{10.26131/IRSA552}; \doi{10.26131/IRSA580}).

% ========================= Bibliography =========================

\begin{thebibliography}{}

\bibitem[Gaia Collaboration et al.(2023)]{gaia_dr3_summary}
Gaia Collaboration, Vallenari, A., et al.\ 2023, \aap, 674, A1, \doi{10.1051/0004-6361/202243940}

\bibitem[Lang(2014)]{unwise_lang2014}
Lang, D.\ 2014, \aj, 147, 108, \doi{10.1088/0004-6256/147/5/108}

\bibitem[Lyke et al.(2020)]{sdss_dr16q}
Lyke, B.~W., et al.\ 2020, \apjs, 250, 8, \doi{10.3847/1538-4365/aba623}

\bibitem[Marocco et al.(2021)]{catwise2020}
Marocco, F., et al.\ 2021, \apjs, 253, 8, \doi{10.3847/1538-4365/abd805}

\bibitem[Meisner et al.(2017)]{unwise_meisner2017}
Meisner, A.~M., Lang, D., \& Schlegel, D.~J.\ 2017, \aj, 153, 38, \doi{10.3847/1538-3881/153/1/38}

\bibitem[Secrest et al.(2022)]{secrest2022}
Secrest, N.~J., von Hausegger, S., Rameez, M., Mohayaee, R., \& Sarkar, S.\ 2022, \apjl, 937, L31, \doi{10.3847/2041-8213/ac88c0}

\end{thebibliography}

\end{document}
